%%%%%%%%%%%%%%%%%%%%%%%%%%%%%%%%%%%%%%%%%%%%%%%%%%%%%%%%%%%%%%%%%%%%%%%%%%%%%%%%
%
%	Mall för enklare LaTeX-dokument
%	Skriven av Viktor Ahlqvist
%	http://www.texempelvis.se
%
%	Tänkt att användas tillsammans med LuaTeX, men bör fungera även i XeTeX
%
%	Notera att jag inte beskriver alla inställningar som finns till paketen, det
%	är alltid en bra ide att läsa paketens manualer för att få reda på vilka 
%	andra inställningar som finns. Manualerna finns dels på CTAN
%	(http://www.ctan.net) eller via `texdoc` om TeX Live är installerat.
%
%	Exempeltext är taget ifrån August Strindbergs Röda rummet, som inte längre
%	omfattas av upphovsrätt utan är allmängods.
%
%%%%%%%%%%%%%%%%%%%%%%%%%%%%%%%%%%%%%%%%%%%%%%%%%%%%%%%%%%%%%%%%%%%%%%%%%%%%%%%%

% Nag är ett paket som bör stå överst i alla dokument. Det tillför inga nya
% funktioner, men skriver varningar om man använder gamla, utdaterade,
% kommandon.
\RequirePackage[l2tabu, orthodox]{nag}

% KOMA-Script är en samling klasser och paket som syftar att ersätta LaTeX
% standardklasser. Förutom att vara snyggare så är de även mer anpassade till
% europeiska konventioner, bland annat så är a4-papper standard.
% 
%	scrartcl motsvarar LaTeXs article-klass
%	scrbook motsvarar book
%	scrreprt motsvarar report
%
% Mer information om vad som skiljer klasserna åt finns att läsa manualen på
% http://mirrors.ctan.org/macros/latex/contrib/koma-script/doc/scrguien.pdf
% Inställningarna till KOMA-Script klasserna kan antingen skickas som 
% alternativa argument i \documentclass eller via funktionen \KOMAoptions.
% Vissa argument bör dock skickas via \documentclass.
\documentclass[%
	%paper=a5,pagesize % Om man vill använda ett annat pappersformat än A4 måste
					% man givetvis säga till om det. pagesize bör användas om 
					% man ändrar pappersstorleken, se sida 43 i manualen.
	%parskip=full,	% Standard i LaTeX är att avgränsa paragrafer med ett indrag
					% Ibland vill man ha mellanrum istället, och det gör parskip
	%twoside,		% Ger dubbelsidigt dokument med alternerande marginaler o.d.
	%twocolumn,		% Ger dokument med två kolumner.	
	]{scrartcl}

% Graphicx är det utökade grafikpaketet från 'graphics bundle'. Lägger till
% en uppsjö kommandon kopplade till importerad grafik, skalning och annat.
\usepackage{graphicx}

% amsmath är en samling nästan nödvändiga paket för att typsätta matematik.
% Här väljer vi dessutom
% tbtags som innebär att för delade ekvationer så placeras numreringen i höjd 
% med sista raden om ekvationsnumren är på höger sida och första raden om numren
% är på vänster sida.
% intlimits som säger att gränserna i integraler ska visas ovanför och under 
% integraltecknet och inte vid sidan om som annars är standrd. Tar lite mer 
% plats i höjdled, men är betydligt snyggare tycker jag.
\usepackage[tbtags,intlimits]{amsmath}

% scrpage2 är en del av KOMA-Script paketet och lägger till enklare och bättre
% möjligheter att justera sidhuvuden och -fötter.
\usepackage{scrpage2}

% Polyglossia är ett paket som lägger till flerspråksstöd i LaTeX. Genom att 
% ladda ett språk i Polyglossia får man dels avstavning för det aktuella 
% språket, dels översättning av rubriker till innehållsförteckning, datum och 
% liknande "standardtext" som skrivs ut av LaTeX.
\usepackage{polyglossia}
% När man har laddat Polyglossia får man dessutom ställa in standardspråket den 
% ska använda i dokumentet. Här väljer vi svenska som huvudspråk och anger 
% dessutom att vi vill ladda brittisk engelska som alternativt språk.
% Engelskan kan sedan användas med \textenglish{This text is in English} eller 
% med \begin{english}This whole section is in English!\end{english}
% Fullständig lista över språk och kommandon för Polyglossia finns i manualen.
\setdefaultlanguage{swedish}
\setotherlanguage[variant=uk]{english}

%%% Typsnitt %%%%%%%%%%%%%%%%%%%%%%%%%%%%%%%%%%%%%%%%%%%%%%%%%%%%%%%%%%%%%%%%%%%
% Det är egentligen inget fel med typsnittet Computer Modern som är standard i 
% LaTeX, men jag tycker personligen inte om det och det, speciellt inte den
% kursiva stilen. Dessutom
% Istället förordar jag typsnittet Linux Libertine, som inte bara är gratis utan
% dessutom öppen källkod. Innehåller i stort sett alla stilar och varianter man
% kan behöva.
\usepackage{libertine}
%%%%%%%%%%%%%%%%%%%%%%%%%%%%%%%%%%%%%%%%%%%%%%%%%%%%%%%%%%%%%%%%%%%%%%%%%%%%%%%%

% Unicode-math, som bara fungerar på LuaTeX och XeTeX är ett paket för att ge
% fullt unicode-stöd i matteläget och även göra det lätt att byta typsnitt.
% Bifogat manualen finns en lång lista över alla tecknen i unicode och hur man 
% infogar dessa på ett lätt sätt. Filen heter unicode-symbols.pdf.
% Unicode-math ändrar dessutom på vissa underligheter hos LaTeX, tex att \bm 
% används för en "bold symbol" medan \mathbf används om det är feta bokstäver 
% man vill åt. Unicode-math ordnar så att \mathbf kan användas för både latinska
% och grekiska bokstäver.
% 
% Jag har även några inställningar som ändrar beteendet hos TeX
%	math-style=ISO	Som standard så använder TeX kursiva gemena grekiska 
%					bokstäver och raka versala sådana. Detta stämmer dåligt 
%					överrens med ISO-standarder som säger att variabler alltid 
%					ska vara kursiva. Den här inställningen ändrar så att både
%					versaler och gemener är kursiva.
%	bold-style=ISO	Även här så stämmer inte TeX:s standardbeteende överrens med
%					ISO:s standarder. TeX tycker nämligen om att göra feta 
%					versaler i en rak stil, vilket även det motsäger 
%					konventionen att variabler skrivs kursivt. bold-style=ISO
%					gör att de feta glyferna från \mathbf fortfarande är 
%					kursiva.
%	vargreek-shape=unicode	Det finns två olika gemena versioner av de grekiska 
%					bokstäverna epsilon och phi. TeX definerar \epsilon och 
%					\varepsilon för att komma åt de två olika varianterna. 
%					TeX:s definition är dock omvänd mot hur dessa symboler är 
%					ordnade i Unicode. Då jag personligen tycker att de 
%					symboler som TeX tycker är varianten är betydligt snyggare 
%					gör den här inställningen att jag slipper använda \varphi 
%					och det dessutom blir mer logiskt för alla som inte är vana 
%					vid TeX.
\usepackage[math-style=ISO,bold-style=ISO,vargreek-shape=unicode]{unicode-math}
% Personligen tycker jag att TeX Gyre Pagella är ett trevligt typsnitt för 
% matematik. Det behöver inte stå en sökväg till en fil, utan om typsnittet är 
% installerat på datorn räcker det med att skriva namnet, exempelvis 
% \setmathfont{Asana Math}. Det går dessutom att använda olika typsnitt för 
% olika delar (t.ex. ett bara för grekiska bokstäver), och detta förklaras i 
% unicode-math:s manual.
% \setmathfont{texgyrepagellamath-regular.otf}
\setmathfont{Asana Math}

% Microtype, "Subliminal refinements towards typographical perfection", är ett
% paket som lägger till stöd för de mikrotypografiska verktyg som pdfTeX
% ursprungligen introducerade, bl.a. 'character protusion' och 'font expansion'.
% (jag vet inte om det finns några svenska termer på detta). Kortfattat så får
% den dokumenten att se snyggare ut genom att göra väldigt små typografiska 
% ändringar. Vi ställer inte in några speciella inställningar utan låter 
% microtype lösa det på bästa sätt. Manualen beskriver dock många funktioner man
% kan finjustera om man vill.
\usepackage{microtype}

% I engelska används punkt som decimalavskiljare och kommatecken för att 
% separera element i listor. Som standard kan alltså kommatecknet inte användas 
% som decimalavskiljare, eftersom mellanrummen blir felaktiga. Icomma löser 
% detta på magiskt sätt med sina 13 rader kod. Om man skriver text på engelska 
% eller använder punkt som avskiljare finns det ingen anledning att ladda 
% paketet och det kan alltså kommenteras bort.
\usepackage{icomma}

% siunitx är ett av mina favoritpaket. Paketets syfte är dels att erbjuda 
% möjligheter till att typsätta enheter, men även kommandot \num för att 
% formatera tal. Manualen rekommenderas verkligen för att förstå alla 
% användbara funktioner (t.ex. avrundning, exponentnotifikation et.c.)
\usepackage{siunitx}
% Istället för att skicka in inställningarna som argument till \usepackage,
% vilket ger samma resultat, används \sisetup (mest för att formatera 
% kommentarer på ett enklare sätt). \sisetup kan dock användas i själva 
% dokumentet för att ändra utskriften för en specifik del och de flesta 
% inställningarna kan även ske individuellt, t.ex. 
% \num[output-complex-root=i]{4+10j}
% Det finns som sagt mängder av inställningar, det här är bara några få. Se 
% till att läsa manualen!
\sisetup{%
	free-standing-units,		% Gör att enheterna från siunitx kan användas 
								% även utanför \SI och \si. Framförallt 
								% användbart för procent, men det kan även vara 
								% trevligt att kunna skriva \metre på 
								% godtyckligt ställe i rapporten.
	exponent-product = \cdot,	% I sverige (och europa?) är det vanligast att 
								% använda en centrerad punkt som 
								% multiplikationssymbol.
    output-decimal-marker={,},	% Kommatecken som decimalavdelare i det 
								% utskrivna talet. Föredras punkt kan raden tas 
								% bort eller tecknet bytas ut.
	output-complex-root = j,	% Om man är elektroteknolog använder man 
								% såklart j istället för i för att representera 
								% roten ur -1. Man kan t.ex. använda 
								% copy-complex-root istället för att få samma 
								% utskrift som man skrivit in i texten.
	complex-root-position = before-number, % Komplexa tal kan antingen skrivas
								% 5+j2 eller 5+2j. Jag föredrar den första 
								% notationen.
	per-mode=symbol,			% Newton per meter kan antingen skrivas N/m 
								% eller Nm^-1. Jag föredrar det första 
								% skrivsättet.
	group-digits=integer,		% Det är vanligt att gruppera tal tre och tre
								% för att förenkla läsandet. Kommandot gör att 
								% så sker, men bara för heltalsdelen.
	group-minimum-digits=4,		% Minsta antalet siffror som krävs för att det 
								% ska grupperas. 4 siffor = tusental.
	group-separator = {\,},		% Styr vilken avgränsare som ska användas vid 
								% grupperingen. Svenska skrivregler (och 
								% ISO-standarder) förespråkar ett 
								% åttondelsfyrkantsmellanrum (thin space).
%	detect-all,					% En inställning som jag inte brukar använda, 
								% men som är vanlig är huruvida siunitx ska 
								% anpassa sin utskrift efter omgivande typsnitt 
								% och inställningar. Standard är att inte bry 
								% sig om det, detect-all medför att den 
								% anpassar allt. Finns fler inställningar som 
								% står i manualen.
	detect-display-math,		% Gör att siunitx känner av om den är i 
								% matteläge och använder dess typsnitt.
	separate-uncertainty = true,% Om talen är angivna med osäkerhet kan man
								% antingen skriva ut det som talet +- 
								% osäkerheten eller med de osäkra siffrorna i 
								% parantes. Kommandot växlar till +- osäkerhet.	
}
% Vad följande rad gör tänker jag inte förklara men hade du läst manualen till 
% siunitx hade det inte varit något problem alls.
% \SI[parse-number = false]{\sqrt{3}}{\metre}

% Tabular, "standardsättet" att göra tabeller har sina begränsningar. Exempelvis
% kan inte tabeller brytas över flera sidor vilket blir ett stort problem vid 
% långa tabeller. Paketet longtable lägger miljön longtable 
% (\begin{longtable}...\end{longtable}) som bland annat gör sidbrytningar, men
% har även flera andra användbara funktioner.
\usepackage{longtable}

% När det ändå pratas om tabeller måste man nästan nämna booktabs, paketet som
% hjälper till att göra tabeller snyggare med en samling användbara kommandon 
% och ordnar avstånd till linjer på ett bra sätt.
% Manualen är dessutom nästan en guide med vad man ska tänka på och hur man ska
% utforma tabeller i sina rapporter.
% Orkar du inte läsa manualen är dock följande två regler det viktigaste
% 1. Never, ever use vertical rules.
% 2. Never use double rules.
\usepackage{booktabs}

% Paketet wrapfig lägger till miljön wrapfigure och wraptable för att få text
% att flyta runt en bild. Mer tänker jag inte skriva utan hänvisar till manualen
% som beskriver dels syntaxen, dels saker man ska tänka på.
%\usepackage{wrapfig}


%%%%%%%%%%%%%%%%%%%%%%%%%%%%%%%%%%%%%%%%%%%%%%%%%%%%%%%%%%%%%%%%%%%%%%%%%%%%%%%%
% Här var det nästan slut på laddning av paket och vi börjar istället göra lite
% andra iställningar och definitioner.

% KOMA-Script lägger till en del egna kommandon för att styra bl.a. typsnitt.
% Jag tycker att KOMAS standardutseendet på rubriker inte är särskilt snyggt,
% men som tur är kan de bytas på ett enkelt sätt.
% disposition byter utseende på alla sektionstitlar, från \part ner till 
% \subparagraph. Vill man bara byta typsnitt i \section kan man skriva t.ex. 
% skriva \setkomafont{section}{\sffamily}.
\setkomafont{disposition}{\sffamily}
% Man kan även byta utseende på sidhuvud och sitfot. Jag tycker om att skriva 
% texten i sidhuvudet med kapitäler och sidfoten med vanlig text, vilket 
% följande två rader gör.
\setkomafont{pagehead}{\normalfont\scshape}
\setkomafont{pagefoot}{\normalfont}

% Jag tycker mer om punkt som multiplikationssymbol istället för det kryss som
% är standard i LaTeX. För att ändå göra det tydligt vad jag menar när jag 
% skriver i matematikläget så brukar jag sonika definera om \times
% unicode-math gör något lurigt så omdefinitionen måste ske efter 
% \begin{document}, men lyckligtvis finns det ett Latex-kommando för just detta.
\AtBeginDocument{\let\times\cdot}

% Det är en bra ide att definera egna kommandon för t.ex. formattering och 
% styra utseendet på ett ställe istället för att hårdkoda in formattering i 
% texten.
% Nedanstående är exempel på kommandon som jag brukar använda. Det översta,
% \acr är en förkortning för acronym och jag brukar använda det för 
% förkortningar. Det kan ibland hända att man vill använda 'riktiga' versaler 
% för förkortningar istället för kapitäler, och då är det enkelt att bara byta 
% en rad på ett ställe
% Ger kapitäler. Precis som alltid så kommer versaler som skrivs in skrivas ut
% som versaler. \acr{Abb} kommer alltså ge versalt 'A' och 'bb' som kapitäler.
\newcommand*{\acr}[1]{\textsc{#1}}
% Här är en variant som först konverterar texten till gemener, vilket betyder
% att i Abb kommer alla bokstäver skrivas ut som kapitäler.
%\newcommand{\acr}[1]{\textsc{\MakeLowercase{#1}}}
% Och sist är en variant för om man hellre vill använda 'riktiga' versaler.
%\newcommand{\acr}[1]{\MakeUppercase{#1}}
% Ovanstående är bara exempel på varför det är så praktiskt att beskriva vad
% man menar i texten istället för att detaljstyra utseende.

% Enligt ISO:s standarder ska d:et i d/dx eller dx i en integral skrivas med 
% upprätt stil och inte med kursiv, som blir fallet om man bara skriver ett
% d i filen. Följande kommando är ett exempel på hur man kan ordna det.
\newcommand*\diff{\mathop{}\!\mathrm{d}}

% Vektorer kan skrivas på många olika sätt, standard i TeX är att skriva de med
% en lite pul över bokstaven. Jag föredrar att istället skriva med feta 
% bokstäver och definerar därför om \vec till en bättre definition.
\renewcommand{\vec}[1]{\ensuremath{\mathbf{#1}}}

% Ett annat exempel på kommando man kan vilja definera själv är ett alternativ
% till +- symbolen. Ofta vill man göra en liten parantes runt t.ex. minustecknet
% om den lösningen inte är möjlig. Följande två kommandon lägger till dels 
% \varpm och \varmp.
\newcommand\varpm{\mathbin{\vcenter{\hbox{%
  \oalign{\hfil$\scriptstyle+$\hfil\cr
          \noalign{\kern-.3ex}
          $\scriptscriptstyle({-})$\cr}%
}}}}
\newcommand\varmp{\mathbin{\vcenter{\hbox{%
  \oalign{$\scriptstyle({+})$\cr
          \noalign{\kern-.3ex}
          \hfil$\scriptscriptstyle-$\hfil\cr}%
}}}}
%%%%%%%%%%%%%%%%%%%%%%%%%%%%%%%%%%%%%%%%%%%%%%%%%%%%%%%%%%%%%%%%%%%%%%%%%%%%%%%%
% Nu ska den metadata som  behövs fyllas i. \title, \author och \date är 
% klassiska LaTeX-kommandon och skrivs ut i dokumentet av \maketitle.
% KOMA-Script lägger dessutom till en del ytterligare kommandon
% \titlehead	Som skrivs ut längst upp på titelsidan
% \subject		Som skrivs ut direkt ovanför \title
% \subtitle		Som skrivs ut under \title
% \publishers	Som skrivs ut sist. Det här kommandot kan givetvis användas
%				till annan text än just publicent. 
% \thanks		Som används för fotnoter kopplat mot författarna (är dock ett
%				standard-LaTeX-kommando).

% Titeln på dokumentet, inte särskilt svårt alls.
\title{Röda rummet}
\subtitle{Skildringar ur artist-- och författarlivet}
% Författare ska separeras med \and för att utseendet ska bli rätt.
\author{August Strindberg\thanks{Vill säkert tacka Siri von Essen} \and Viktor Ahlqvist\thanks{Som på något sätt vill få lite ära för det här dokumentet i alla fall}}
% Och som vanligt gäller att om \date utelämnas används dagens datum.

% Sidhuvuden och -fötter sätts automatiskt, men det är inte alltid man vill ha
% det utseendet. Paketet scrpage2 som laddades ovan ger bra möjligheter att 
% styra innehållet själv.
%
% Rensar sidhuvud och fot, är bra om vi vill byta ut all information i dessa.
\clearscrheadfoot
% \ofoot, outside foot. I ett enkelsidigt dokument blir det alltid högsida, men 
% i ett dubbelsidigt kommer detta givetvis växla.
% Det valfria argumentet [\pagemark] är det som används vid scrpage2:s 
% 'plain'-stil, som används på bl.a. titelsidan (där man ofta inte vill ha 
% sidhuvud).
\ofoot[\pagemark]{\pagemark}
% \ihead Inside head. De andra två sektionerna är alltså \chead och \ohead.
\ihead{exempelmall i latex}
\ohead{viktor ahlqvist}

% LaTeX måste få veta vilken stil den ska använda. Kan antingen sättas här i 
% preamble:n eller i dokumentet (och kan även ändras i olika delar av 
% dokumentet).
\pagestyle{scrheadings}

%%%%%%%%%%%%%%%%%%%%%%%%%%%%%%%%%%%%%%%%%%%%%%%%%%%%%%%%%%%%%%%%%%%%%%%%%%%%%%%%
% Ett sista paket ska dock laddas innan vi börjar skriva på dokumentet. 
% hyperref är ett paket som lägger till möjligheten till interna och externa 
% länkar i pdf-dokumenten. Som standard gör den t.ex. innehållsförteckningen 
% till länkar. Har över 100 inställningar som alla givetvis står i manualen.
%
% hyperref behöver veta vilka andra paket som är laddade och det är därför 
% viktigt att det laddas sist. Det finns vissa program som ska laddas efteråt, 
% men det är få och det brukar i så fall stå i deras dokumentation.
% 'hidelinks' gör att länkarna i dokumentet inte markeras på något speciellt
% sätt. Personligen tycker att det är störande när länkar skrivs i en annan 
% färg, men det gör det samtidigt tydligare att det är länkar.
\usepackage[hidelinks]{hyperref}
%
% Hyperref ger även möjligheten att sätta pdf-metadata, vilket är användbart 
% speciellt om filen ska skickas till någon databas. Nedanstående kod fyller i
% den titel och författare som sattes ovan. Då dessa kommandon är 'skyddade' 
% med @ för att bara kunna ändras i dokumentklassen behöver vi byta funktion på 
% @ med \makeatletter.
\let\oand\and
\def\and{, }
\makeatletter
\hypersetup{
  unicode,					% För att få unicode-kodade pdfsträngar.
  breaklinks,				% Tillåter att länkar radbryts.
  pdfauthor = {\@author},	% Sätter författare på dokumentet.
  pdftitle = {\@title},		% Sätter titel på dokumentet
  pdfsubject = {\@title},	% Ämne  
}
\makeatother
\let\and\oand


%%%%%%%%%%%%%%%%%%%%%%%%%%%%%%%%%%%%%%%%%%%%%%%%%%%%%%%%%%%%%%%%%%%%%%%%%%%%%%%%
% Här börjar det faktiska innehållet till dokumentet.
\begin{document}
\maketitle 			% Skriver ut titeldatan
\tableofcontents	% Skriver ut innehållsförteckning

\section{Stockholm i fågelperspektiv}
Det var en afton i början av maj. Den lilla trädgården på Mosebackehade ännu 
icke blivit öppnad för allmänheten och rabatterna voro ej uppgrävda; 
snödropparne hade arbetat sig upp genom fjolårets lövsamlingar och höllo just 
på att sluta sin korta verksamhet för att lämna plats åt de ömtåligare 
saffransblommorna, vilka tagit skydd under ett ofruktsamt päronträd; syrenerna 
väntade på sydlig vind för att få gå i blom, men lindarne bjödo ännu 
kärleksfilter i sinao brustna knoppar åt bofinkarne, som börjat bygga sina 
lavklädda bon mellan stam och gren; ännu hade ingen mänskofot trampat 
sandgångarne sedan sista vinterns snö gått bort och därför levdes ett obesvärat 
liv därinne av både djur och blommor. Gråsparvarne höllo på att samla uppskräp, 
som de sedan gömde under takpannorna på navigationsskolans hus; de drogos om 
spillror av rakethylsor från sista höstfyrverkeriet, de plockade halmen från 
unga träd som året förut sluppit ur skolan på Rosendal -- och allting sågo de! 
De hittade barège-lappar i bersåer och kunde mellan stickorna på en bänkfot 
draga fram hårtappar efter hundar, som icke slagits där sedan Josefinadagen i 
fjor. Där var ett liv och ett kiv.

\section{Bröder emellan}
Linkramhandlaren Carl Nicolaus Falk, son till avlidne linkramhandlaren, en av 
Borgerskapets femtio äldste och kaptenen vidBorgerskapets infanteri, Kyrkorådet 
och Ledamoten av direktionen förStockholms Stads Brandförsäkringskontor Herr 
Carl Johan Falk -- och bror till förre e.o. notarien numera litteratören Arvid 
Falk, hade sin affär, eller, som hans ovänner helst kallade den, bod, vid 
Österlånggatan, så snett emot Ferkens gränd, att bodbetjänten kunde,när han 
tittade upp från sin roman, som han satt och fuskade med under disken, se ett 
stycke av en ångbåt, ett hjulhus, en klyvarbom eller så, och en trätopp på 
Skeppsholmen samt en bit luft ovanför. Bodbetjänten som lydde det icke ovanliga 
namnet Andersson,och han hade lärt lyda, hade nu på morgonen öppnat, hängt ut 
en lintott, en ryssja, en ålmjärde, en knippa metspön samt en katseospritad 
fjäder; därpå hade han sopat boden, och strött sågspån på golvet samt slagit 
sig ner bakom disken, där han av en tom ljuslåda tillrett en slags råttfälla 
som han gillrat upp med en stångkrok och i vilken hans roman ögonblickligen 
kunde falla, om patron eller någon av dennes bekanta skulle inträda. Några 
kunder tycktes han ej befara, dels emedan det var tidigt på morgonen, dels 
emedan han icke var van vid överflöd på sådana. Affären hade blivit grundlagd 
under Salig Kung Fredriks dagar -- Carl Nicolaus Falk hade liksom allt annat 
även ärvt detta uttryck från sin far, vilken åter fått det i rätt nedstigande 
linje från sin farfar -- den hade blomstrat och givit ifrån sig bra med pengar 
förr, ända till för några år sedan, då det olycksaliga 
``representationsförslaget'' kom och gjorde slut på all handel,förstörde alla 
utsikter, hämmade all företagsamhet och hotade att bringa Borgerskapet till 
undergång. Så uppgav Falk själv, men andra menade att affären icke sköttes och 
att en svår konkurrent etablerat sig nere vid slussplan. Falk talade dock icke 
i onödan om affärens lägervall och han var en tillräckligt klok man att välja 
både tillfälle och åhörare när han slog an den strängen. När någon av hans 
gamla affärsvänner framställde en vänlig förundran över den minskade trafiken, 
då talade han om, att han byggde på sin partihandel på landsorten och bara hade 
boden som en skylt, och de trodde honom, ty han hade ett litet kontor innanför 
boden, där han mest vistades, närhan icke var ute i stan eller på Börsen. Men 
när hans bekanta -- det var något annat det -- det var notarien och magistern 
-- yttrade samma vänliga oro -- då var det de dåliga tiderna, härledande sig 
från representationsförslaget - som vållade stagnationen.

Emellertid hade Andersson, vilken blivit störd av några pojkar som i dörren 
frågat vad metspöna kostat, råkat att titta ut åt gatan och fått sikte på unge 
herr Arvid Falk. Som han fått låna boken av just honom fick den bli liggande 
där den låg och det var med en ton av förtrolighet och en min av hemligt 
förstånd han hälsade sin förre lekkamrat då denne inträdde i boden.\\


\noindent -- Är han där uppe? frågade Falk med en viss oro.\\
-- Han håller på att lösa följande problem sa Andersson, ty han hade insett att 
från kortslutningsprovets data kan $X_k$ och $R_k$ bestämmas.
\begin{gather*}
	I_k = I_{N1} = \frac{S_N}{\sqrt{3} U_{N1}} = \SI{10,94}{A}\\
	R_k = \frac{P_k}{3I_k^{\phantom{k}2}} = \SI{3,70}{\ohm/fas}\\
	X_k = \sqrt{Z_k^{\phantom{k}2} - R_k^{\phantom{k}2}} = \left\{ Z_k =
	\frac{U_{k}}{\sqrt{3}I_k} \right\} = \SI{15,94}{\ohm/fas}
	\intertext{Spänningsfallet $\upDelta  U$ kan nu beräknas för att sedan 
	användas för att beräkna $U_2$. Om spänningsfallet är litet kan 
	$\vec{U}_{\mathrm{tvärf}}$ försummas.}
	\upDelta  U = \sqrt{3}I_2^{\,\prime} \left( R_k \cos{\phi} + X_k \cos{\phi} 
	\right) = 1,32I_2\si{V}\\
	\implies U_2 = \frac{U_{N1} - \upDelta  U}{\frac{6600}{400}} = 
	\left(400-0,08I_2\right)\si{V}
	\intertext{Kopparförlusterna $P_{\mathrm{Cu}}$ beräknas och används sedan 
	för att beräkna den inmatade effekten.}
	P_{\mathrm{Cu}} = 3R_k \left( I_2^{\,\prime}\right)^2\\
	P_{\mathrm{1}} = P_{\mathrm{Cu}} + P_0 + P_2 = \left\{ P_2 = \sqrt{3} U_2 
	I_2 \cos{\phi} \right\} = \left(-0,07I_2^{\phantom{2}2} + 554,26I_2 + 
	1050\right)\si{W}
	\intertext{Verkningsgraden $\eta$ definieras som kvoten mellan uttagen 
	effekt och inmatad effekt.}
	\eta = \frac{P_2}{P_1} = 
	\frac{-0,11I_2^{\phantom{2}2}+554,26I_2}{-0,07I_2^{\phantom{2}2} + 
	554,26I_2 + 1050}
	\intertext{Den maximala verkningsgraden fås genom att derivera $\eta$ och 
	sedan lösa ut $I_2$.}
	0 = \frac{\diff \eta}{\diff I_2} = \frac{-22,1704 I_2^{\phantom{2}2} - 
	231I_2 + 581 973}{\left(-0,07I_2^{\phantom{2}2} +554,26I_2 +1050 
	\right)^2}\\
	\implies I_2 = \frac{25\left(-825 \varpm 
	\sqrt{\num{658975227}}\right)}{3959} = \SI{156,89}{A}
	 \intertext{Med $I_2$ känd kan den maximala verkningsgraden bestämmas.}
	 \eta_{\mathrm{max}} = 
	 \frac{-0,11\times156,89^2+554,26\times156,89}{-0,07\times156,89^2 + 
	 554,26\times156,89 + 1050} = 0,98
	 \intertext{Den uttagna aktiva effekten vid $\eta_{\mathrm{max}}$ kan nu 
	 beräknas.}
	 P_{2\eta} = -0,11\times156,89^2+554,26\times156,89 = \SI{84,25}{kW}
\end{gather*}

\end{document}